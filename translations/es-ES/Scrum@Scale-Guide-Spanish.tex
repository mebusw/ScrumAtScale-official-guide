% Generated by GrindEQ Word-to-LaTeX 
\documentclass{article} %%% use \documentstyle for old LaTeX compilers

\usepackage[english]{babel} %%% 'french', 'german', 'spanish', 'danish', etc.
\usepackage{amssymb}
\usepackage{amsmath}
\usepackage{txfonts}
\usepackage{mathdots}
\usepackage[classicReIm]{kpfonts}
\usepackage[dvips]{graphicx} %%% use 'pdftex' instead of 'dvips' for PDF output

% You can include more LaTeX packages here 


\begin{document}

%\selectlanguage{english} %%% remove comment delimiter ('%') and select language if required


\noindent 

\noindent 

\noindent 

\noindent 

\noindent 
\[18\] 
\[1\] 
\textbf{La Gu\'{i}a de Scrum@Scale{\circledR}}

\noindent \textbf{La gu\'{i}a definitiva de Scrum@Scale:}

\noindent \textbf{Escalabilidad que funciona}

\noindent \textbf{\includegraphics*[width=3.29in, height=3.81in, keepaspectratio=false]{image12}}

\noindent \textbf{}

\noindent Versi\'{o}n 1.02 -- 21 Agosto 2018

\noindent Traducida al castellano - Octubre 2018 

\noindent por Ana Bardoneschi y Paula Kvedaras

\noindent \textbf{}

\noindent {\copyright}1993-2018 Jeff Sutherland and Scrum Inc., Todos los derechos reservados Scrum@Scale es una marca registrada de Scrum Inc.Lanzado bajo licencia Creative Commons 4.0 Attribution-Sharealike

\noindent \eject 

\noindent 

\noindent Prop\'{o}sito de la gu\'{i}a Scrum@Scale 3¿Por qu\'{e} Scrum@Scale? 3Definici\'{o}n de Scrum@Scale 4Componentes del Framework Scrum@Scale{\circledR} 5Cultura impulsada por valores 5Comenzar con Scrum@Scale 6Ciclo de Scrum Master 7Proceso a nivel de equipo 7Coordinar el "C\'{o}mo" - El Scrum de Scrums 7El Scrum de Scrums Master (SoSM) 8Escalar el SoS 8El Equipo de Acci\'{o}n Ejecutiva 9Backlog y responsabilidades del EAT 10Salidas / Resultados del Ciclo de Scrum Master 11Ciclo del Product Owner 12Coordinar el "Qu\'{e}" -  El MetaScrum 12El Chief Product Owner (CPO) 13Escalar el MetaScrum 13El MetaScrum Ejecutivo (EMS) 14Salidas / Resultados de la Organizaci\'{o}n de Product Owners 14Conectar los ciclos de PO/SM 15Entender la Retroalimentaci\'{o}n 15M\'{e}tricas y Transparencia 16Algunas notas sobre el dise\~{n}o organizacional 16Nota final 18Agradecimientos 18

\noindent \textbf{}

\noindent \textbf{}

\noindent \eject \textbf{}

\noindent 
\section{Prop\'{o}sito de la gu\'{i}a Scrum@Scale}

\noindent 

\noindent Scrum, como se describi\'{o} originalmente en la Gu\'{i}a de Scrum, es un \textit{framework }para que el desarrollo, la entrega y el mantenimiento de productos complejos sean realizados por un solo equipo. Desde su comienzo, el uso se ha extendido a la creaci\'{o}n de productos, procesos, servicios y sistemas que requieren el trabajo de varios equipos. Scrum@Scale fue creado para coordinar de manera eficiente este nuevo ecosistema de equipos de tal manera que optimice la estrategia general de la organizaci\'{o}n. Se alcanza este objetivo estableciendo una "m\'{i}nima burocracia viable" a trav\'{e}s de una arquitectura aplicable a toda escala, que naturalmente extiende el funcionamiento de un equipo Scrum a toda la organizaci\'{o}n.

\noindent 

\noindent Esta gu\'{i}a contiene las definiciones de los componentes del \textit{framework }de Scrum@Scale incluyendo sus roles escalados, eventos escalados y artefactos empresariales, as\'{i} como las reglas que los unen.

\noindent 

\noindent El Dr. Jeff Sutherland desarroll\'{o} Scrum@Scale basado en los principios fundamentales tras Scrum, la teor\'{i}a de Sistemas Adaptativos Complejos, la teor\'{i}a de juegos y la tecnolog\'{i}a orientada a objetos. Esta gu\'{i}a fue desarrollada con el aporte de muchos practicantes de Scrum bas\'{a}ndose en los resultados de su trabajo de campo. El objetivo de esta gu\'{i}a es que el lector pueda implementar Scrum@Scale por s\'{i} mismo.\textbf{}

\noindent 
\subsection{¿Por qu\'{e} Scrum@Scale?}

\noindent 

\noindent Scrum fue dise\~{n}ado para que un solo equipo pueda trabajar a su \'{o}ptima capacidad mientras mantiene un ritmo sostenible. En la pr\'{a}ctica, se encontr\'{o} que a medida que aumentaba la cantidad de equipos Scrum dentro de una organizaci\'{o}n, la producci\'{o}n (producto funcionando) y la velocidad de esos equipos comenzaba a disminuir (debido a problemas como las dependencias entre equipos y la duplicaci\'{o}n de trabajo). Qued\'{o} en evidencia que era necesario un \textit{framework }para coordinar eficazmente esos equipos y as\'{i} lograr escalabilidad lineal. Scrum@Scale est\'{a} dise\~{n}ado para lograr este objetivo a trav\'{e}s de su arquitectura que escala libremente.

\noindent 

\noindent Al utilizar una arquitectura aplicable a toda escala, la organizaci\'{o}n no est\'{a} obligada a crecer de una manera particular determinada por un conjunto de reglas arbitrarias; en cambio puede crecer org\'{a}nicamente seg\'{u}n sus necesidades \'{u}nicas y al ritmo sostenible de cambio que puede ser aceptado por los grupos de individuos que componen la organizaci\'{o}n. La simplicidad del modelo de Scrum@Scale es esencial para una arquitectura aplicable a toda escala y evita cuidadosamente introducir complejidad adicional que causar\'{i}a que la productividad de cada equipo disminuyera a medida que se crean m\'{a}s equipos.

\noindent 

\noindent Scrum@Scale est\'{a} dise\~{n}ado para escalar a trav\'{e}s de la organizaci\'{o}n como un todo: todos los departamentos, productos y servicios. Se puede aplicar a trav\'{e}s de varias \'{a}reas en todo tipo de organizaciones en la industria, el gobierno o la educaci\'{o}n.

\noindent 
\subsection{Definici\'{o}n de Scrum@Scale}

\noindent 

\noindent Scrum: Es un \textit{framework }dentro del cual las personas pueden abordar problemas complejos adaptativos, mientras entregan de manera productiva y creativa productos del mayor valor posible.

\noindent 

\noindent La Gu\'{i}a Scrum es el m\'{i}nimo conjunto de caracter\'{i}sticas que permiten la inspecci\'{o}n y la adaptaci\'{o}n a trav\'{e}s de la transparencia radical para impulsar la innovaci\'{o}n, la \textit{performance }y la felicidad del equipo. 

\noindent 

\noindent Scrum@Scale: Es un \textit{framework }dentro del cual redes de equipos Scrum que operan consistentemente con la Gu\'{i}a de Scrum pueden abordar  problemas complejos adaptativos, mientras entregan de manera creativa productos del mayor valor posible. 

\noindent 

\noindent \textbf{NOTA}: Estos "productos" pueden ser hardware, software, sistemas integrados complejos,procesos, servicios, etc., dependiendo del \'{a}rea de los equipos Scrum.

\noindent 

\noindent Scrum@Scale es:

\begin{enumerate}
\item  Ligero - la m\'{i}nima burocracia viable

\item  F\'{a}cil de entender - consiste solo en equipos Scrum.

\item  Dif\'{i}cil de dominar - requiere implementar un nuevo modelo operacional.
\end{enumerate}

\noindent 

\noindent Scrum@Scale es un \textit{framework }para escalar Scrum. Simplifica radicalmente la escalabilidad usando Scrum para escalar Scrum. 

\noindent 

\noindent En Scrum, se tiene cuidado de separar la responsabilidad del "qu\'{e}" de la del "c\'{o}mo".  Se tiene el mismo cuidado en Scrum@Scale para que el \'{a}mbito y la responsabilidad sean expresamente entendidos para eliminar el conflicto organizacional innecesario que impide que los equipos logren su productividad \'{o}ptima.

\noindent 

\noindent Scrum@Scale consiste en componentes que permiten que una organizaci\'{o}n personalice su estrategia de transformaci\'{o}n e implementaci\'{o}n. Les da la capacidad de apuntar sus esfuerzos incrementales de transformaci\'{o}n en las \'{a}reas que consideran m\'{a}s valiosas o m\'{a}s necesitadas de cambiar y luego progresar sobre otras.

\noindent 

\noindent Para separar estos dos \'{a}mbitos, Scrum@Scale contiene dos ciclos: el Ciclo de Scrum Master (el "c\'{o}mo") y el Ciclo de Product Owner (el "qu\'{e}"), conectados en dos puntos. En conjunto, estos dos ciclos producen un \textit{framework }poderoso para coordinar los esfuerzos de m\'{u}ltiples equipos a lo largo de un \'{u}nico camino.\textbf{\underbar{}}

\noindent \textbf{\underbar{}}

\noindent 
\subsection{}

\noindent 
\subsection{}

\noindent 
\subsection{}

\noindent 
\subsection{Componentes del Framework Scrum@Scale{\circledR}}

\noindent \textbf{\underbar{}}

\noindent \includegraphics*[width=6.28in, height=3.63in, keepaspectratio=false]{image13}

\noindent 
\subsection{}

\noindent 
\subsection{Cultura impulsada por valores}

\noindent 

\noindent Adem\'{a}s de separar la responsabilidad del "qu\'{e}" y el "c\'{o}mo", Scrum@Scale tiene como objetivo construir organizaciones saludables creando una cultura impulsada por valores en un entorno emp\'{i}rico. Los valores de Scrum son: Apertura, Coraje, Enfoque, Respeto y Compromiso. Estos valores promueven la toma de decisiones emp\'{i}ricas, que dependen de los tres pilares Transparencia, Inspecci\'{o}n y Adaptaci\'{o}n.

\noindent 

\noindent La Apertura promueve la transparencia en todo el trabajo y todos los procesos, sin los cuales no hay capacidad de inspeccionarlos honestamente ni de intentar su adaptaci\'{o}n para mejorarlos. El Coraje se refiere a hacer los saltos audaces necesarios para entregar valor m\'{a}s r\'{a}pido de forma innovadora.

\noindent 

\noindent El Enfoque y el Compromiso se refieren a la forma en que gestionamos nuestras obligaciones laborales, poniendo la entrega de valor al cliente como la mayor prioridad. Por \'{u}ltimo, todo esto debe ocurrir en un ambiente basado en el Respeto por los individuos que hacen el trabajo, ya que sin ellos no puede crearse nada.

\noindent 

\noindent Scrum@Scale ayuda a las organizaciones a prosperar apoyando un modelo de liderazgo transformacional que fomenta un entorno positivo para trabajar a un ritmo sostenible y priorizar el compromiso de entregar valor al cliente.

\noindent 
\subsection{}

\noindent 
\subsection{Comenzar con Scrum@Scale}

\noindent 

\noindent Al implementar grandes redes de equipos, es cr\'{i}tico desarrollar un \textbf{Modelo de Referencia} escalable para un peque\~{n}o conjunto de equipos. Cualquier deficiencia en una implementaci\'{o}n Scrum se amplificar\'{a} cuando se involucren varios equipos. Muchos de los problemas iniciales del escalamiento ser\'{a}n las pol\'{i}ticas y los procedimientos de la organizaci\'{o}n o las pr\'{a}cticas de desarrollo que bloquean la alta \textit{performance }y frustran a los equipos.

\noindent 

\noindent Por lo tanto, el primer desaf\'{i}o es crear un peque\~{n}o conjunto de equipos que implementen bien Scrum. Esto se logra de mejor manera creando un \textbf{Equipo de Acci\'{o}n Ejecutiva} (\textbf{EAT}, por sus siglas en ingl\'{e}s), responsable del desarrollo y la ejecuci\'{o}n de la estrategia de transformaci\'{o}n. El EAT debe estar compuesto por personas empoderadas pol\'{i}tica y financieramente para garantizar la existencia del Modelo de Referencia. Este conjunto de equipos trabaja en resolver los \textit{issues }de la organizaci\'{o}n que bloquean la agilidad y crean un Modelo de Referencia para Scrum que se sepa que funciona en la organizaci\'{o}n y pueda ser usado como un patr\'{o}n para escalar Scrum en toda la organizaci\'{o}n.

\noindent 

\noindent A medida que el Modelo de Referencia de los equipos acelera, se evidencian impedimentos y cuellos de botella que retrasan la entrega, producen desperdicios o impiden  la agilidad del negocio. La forma m\'{a}s eficaz y efectiva de eliminar estos problemas es extender Scrum a toda la organizaci\'{o}n, as\'{i} toda la cadena de valor se optimiza.

\noindent 

\noindent Scrum@Scale logra una escalabilidad lineal en la productividad saturando la organizaci\'{o}n con Scrum y distribuyendo velocidad y calidad org\'{a}nicamente, de forma congruente con la estrategia, los productos y servicios espec\'{i}ficos de la organizaci\'{o}n. 

\noindent 

\noindent 
\section{\eject }

\noindent 
\section{Ciclo de Scrum Master}

\noindent 
\subsection{Proceso a nivel de equipo}

\noindent 

\noindent El \textbf{Proceso a nivel de equipo} constituye el primer punto de contacto entre los ciclos de Scrum Master y Product Owner, y se expresa claramente en la Gu\'{i}a de Scrum. Est\'{a} compuesto de tres artefactos, cinco eventos y tres roles.

\noindent 

\noindent Los objetivos del proceso a nivel de equipo son:

\begin{enumerate}
\item  maximizar el flujo de trabajo completado y con calidad probada,

\item  aumentar la perfomance del equipo con el tiempo,

\item  operar de manera que sea sostenible y enriquecedora para el equipo y

\item  acelerar el ciclo de retroalimentaci\'{o}n\textit{ }del cliente.
\end{enumerate}

\noindent 
\paragraph{Coordinar el "C\'{o}mo" - El Scrum de Scrums}

\noindent 

\noindent Un conjunto de equipos que tienen la necesidad de coordinarse entre s\'{i} componen un "\textbf{Scrum de Scrums}" (\textbf{SoS}, por sus siglas en ingl\'{e}s). Este equipo es un Equipo Scrum en s\'{i} mismo, responsable de un conjunto totalmente integrado de incrementos de producto potencialmente implementables al final de cada Sprint de todos los equipos participantes. Un SoS funciona como un Equipo de Release y debe ser capaz de entregar valor directamente a los clientes. Para hacerlo efectivamente, debe ser consistente con la Gu\'{i}a de Scrum; es decir, tener sus propios roles, artefactos y eventos:

\noindent 

\noindent Roles:

\noindent El SoS necesita tener todas las habilidades necesarias para entregar un producto totalmente integrado potencialmente implementable al final de cada Sprint. (Puede necesitar arquitectos experimentados, l\'{i}deres de QA y otros conjuntos de habilidades operacionales). Adem\'{a}s, tiene representaci\'{o}n del Product Owner para resolver problemas de priorizaci\'{o}n. El Scrum Master del Scrum de Scrums se denomina \textbf{Scrum de Scrums Master (SoSM}, por sus siglas en ingl\'{e}s).

\noindent Eventos:

\noindent El SoSM deber\'{i}a facilitar el Refinamiento del Backlog, evento donde los impedimentos son identificados como "listos" para ser removidos, y el equipo determina cu\'{a}l es la mejor manera de removerlos, y c\'{o}mo sabr\'{a}n cuando est\'{e}n "hechos". Se debe prestar especial atenci\'{o}n a la Retrospectiva de SoS en la que los representantes de los equipos comparten cualquier aprendizaje o proceso de mejora que haya sido exitoso en sus equipos individuales, para estandarizar esas pr\'{a}cticas en todos los equipos dentro del SoS. Dado que el equipo SoS necesita ser capaz de responder en tiempo real a los impedimentos planteados por los equipos participantes, al menos un representante de cada equipo (usualmente el Scrum Master del equipo)  debe participar en la \textbf{Scaled Daily Scrum} (\textbf{SDS}). 

\noindent 

\noindent El evento SDS es similar a la Daily Scrum, ya que optimiza la colaboraci\'{o}n y performance de la red de equipos. Cualquier persona y cantidad de miembros de los equipos participantes pueden asistir seg\'{u}n sea necesario.

\noindent 

\noindent Adicionalmente, el SDS:

\noindent 

\begin{enumerate}
\item  es un \textit{timebox }de 15 minutos o menos,

\item  debe participar un representante de cada equipo incluyendo el equipo de Product Owners.,

\item  es un foro donde los participantes discuten qu\'{e} est\'{a} yendo bien, qu\'{e} se est\'{a} haciendo, y c\'{o}mo los equipos pueden trabajar juntos m\'{a}s efectivamente. Algunos ejemplos de lo que se podr\'{i}a discutir son:

\begin{enumerate}
\item  ¿Qu\'{e} impedimentos tiene mi equipo que les impedir\'{a}n cumplir su Objetivo del Sprint (o impactar\'{a}n el pr\'{o}ximo \textit{release})?

\item  ¿Mi equipo est\'{a} haciendo algo que evitar\'{a} que otro equipo cumpla su Objetivo del Sprint (o impactar\'{a} en el pr\'{o}ximo \textit{release})?

\item  ¿Hemos descubierto algunas nuevas dependencias entre los equipos o alguna forma de resolver una dependencia existente?

\item  ¿Qu\'{e} mejoras hemos descubierto que puedan ser aprovechadas por los equipos?
\end{enumerate}
\end{enumerate}

\noindent 
\paragraph{El Scrum de Scrums Master (SoSM)}

\noindent 

\noindent El Scrum de Scrums Master (SoSM) es responsable del release del esfuerzo conjunto de los equipos y debe:

\begin{enumerate}
\item  hacer visible el  progreso,

\item  hacer visible el \textit{backlog }de impedimentos a toda la organizaci\'{o}n,

\item  eliminar impedimentos que los equipos no pueden resolver por s\'{i} mismos,

\item  facilitar la priorizaci\'{o}n de los impedimentos, poniendo especial atenci\'{o}n en las dependencias entre  equipos y la distribuci\'{o}n del backlog,

\item  mejorar la eficacia del Scrum de Scrums,

\item  trabajar cerca de los Product Owners para tener un Incremento de producto potencialmente implementable al menos cada Sprint y

\item  coordinar la implementaci\'{o}n de los equipos con los Planes de Release de los Product Owners.
\end{enumerate}

\noindent 
\paragraph{Escalar el SoS}

\noindent 

\noindent Dependiendo del tama\~{n}o de la organizaci\'{o}n o de la implementaci\'{o}n, se puede necesitar m\'{a}s de un SoS para entregar un producto muy complejo. En esos casos, se puede crear un \textbf{Scrum de Scrum de Scrums} (SoSoS, por sus siglas en ingl\'{e}s) a partir de m\'{u}ltiples Scrums de Scrums. El SoSoS es un patr\'{o}n org\'{a}nico de equipos Scrum infinitamente escalable. Cada SoSoS debe tener SoSoSM y versiones escaladas de cada artefacto y evento.

\noindent 

\noindent Escalar los SoS reduce la cantidad de canales de comunicaci\'{o}n dentro de la organizaci\'{o}n para que la complejidad est\'{e} encapsulada. El SoSoS interact\'{u}a con un SoS de la misma manera que un SoS interact\'{u}a con un solo equipo Scrum permitiendo la escalabilidad lineal.

\noindent 

\noindent 

\noindent 

\noindent Diagramas de ejemplo: 

\noindent 

\begin{tabular}{|p{2.1in}|p{2.1in}|} \hline 
\includegraphics*[width=2.72in, height=2.97in, keepaspectratio=false, trim=0.43in 0.00in 3.13in 0.00in]{image14}\newline SoS de 5 Equipos & \includegraphics*[width=2.67in, height=2.97in, keepaspectratio=false, trim=3.46in 0.00in 0.15in 0.00in]{image15}\newline SoSoS de 25 Equipos \\ \hline 
\end{tabular}



\noindent \textbf{NOTA}: Si bien la Gu\'{i}a de Scrum define que el tama\~{n}o \'{o}ptimo de un equipo es de 3 a 9 personas, Harvard Research determin\'{o} que el tama\~{n}o \'{o}ptimo de un equipo es de 4,6 personas\footnote{\ Richard\ Hackman,\ Leading\ Teams:\ Setting\ the\ Stage\ for\ Great\ Performances,\ (Boston,\ Harvard\ Business\ School\ Press,\ 2002).}. Experimentos con equipos Scrum de alta \textit{performance }han demostrado reiteradas veces que 4 o 5 personas trabajando es el tama\~{n}o \'{o}ptimo. Es esencial para la escalabilidad lineal que este patr\'{o}n sea el mismo para el n\'{u}mero de equipos en un SoS. Por lo tanto, en los diagramas anteriores y siguientes, se seleccionaron pent\'{a}gonos para representar a un equipo de 5. Estos diagramas son solo ejemplos, su diagrama organizacional puede ser muy diferente.

\noindent 
\subsection{El Equipo de Acci\'{o}n Ejecutiva }

\noindent 

\noindent El Scrum de Scrums para toda la organizaci\'{o}n Agile se llama \textbf{Equipo de Acci\'{o}n Ejecutiva  }(\textbf{EAT}, por sus siglas en ingl\'{e}s). El equipo de liderazgo crea una burbuja Agile en la organizaci\'{o}n donde el Modelo de Referencia opera con sus propias pautas y procedimientos que se integran efectivamente con cualquier parte de la organizaci\'{o}n que no sea Agile. Posee el ecosistema Agile, implementa los valores de Scrum y asegura que los roles de Scrum sean creados y soportados.

\noindent 

\noindent El EAT es el destino final de los impedimentos que no pueden ser eliminados por los SoS que los nutren. Por lo tanto, debe estar integrado por individuos que est\'{e}n empoderados, pol\'{i}tica y financieramente, para eliminarlos. La funci\'{o}n del EAT es coordinar m\'{u}ltiples SoS (o SoSoS) e interactuar con las partes no-Agile de la organizaci\'{o}n. Al igual que cualquier equipo Scrum, necesita un PO y ser\'{i}a mejor si el EAT se re\'{u}ne diariamente como un equipo Scrum. Ellos deben reunirse al menos una vez por Sprint y tener un \textit{backlog }transparente.

\noindent Diagrama que ejemplifica un EAT que coordina 5 grupos de 125 equipos

\noindent 

\noindent \includegraphics*[width=5.00in, height=4.71in, keepaspectratio=false]{image16}

\noindent 

\noindent 

\noindent 
\paragraph{Backlog y responsabilidades del EAT }

\noindent 

\noindent Scrum es un sistema operacional \'{a}gil que es diferente de la gesti\'{o}n de proyectos tradicional. Toda la organizaci\'{o}n SM reporta al EAT, que es responsable de implementar este sistema operacional \'{a}gil estableciendo, manteniendo y mejorando la implementaci\'{o}n en la organizaci\'{o}n. El rol del EAT es crear un Backlog de Transformaci\'{o}n Organizacional (una lista priorizada de las iniciativas \'{a}giles que necesitan llevarse a cabo) y controlar que se lleven a cabo. Por ejemplo, si hay un Ciclo de Vida de Desarrollo de Producto tradicional en la antigua organizaci\'{o}n, es necesario crear, implementar y soportar un nuevo Ciclo de Vida de Desarrollo de Producto \'{a}gil. T\'{i}picamente responder\'{a} mejor a problemas de calidad y de \textit{compliance} que el anterior m\'{e}todo pero ser\'{a} implementado de una manera diferente con diferentes reglas y pautas. El EAT garantiza que se cree y financie una organizaci\'{o}n de Product Owner y que esta organizaci\'{o}n est\'{e} representada en el EAT para apoyar estas tareas.  

\noindent 

\noindent 

\noindent 

\noindent \eject 

\noindent El EAT es responsable de la calidad de Scrum en la organizaci\'{o}n. Sus responsabilidades incluyen, pero no est\'{a}n limitadas a:

\begin{enumerate}
\item  crear un sistema operacional \'{a}gil para el Modelo de Referencia a medida que escala en la organizaci\'{o}n, incluyendo reglas operacionales, procedimientos y lineamientos para hacer posible la agilidad,

\item  medir y mejorar la calidad de Scrum en la organizaci\'{o}n,

\item  desarrollar capacidades dentro de la organizaci\'{o}n para conseguir agilidad empresarial,

\item  crear un centro de aprendizaje continuo para profesionales Scrum, y

\item  apoyar la exploraci\'{o}n de nuevas formas de trabajo.
\end{enumerate}

\noindent 

\noindent Finalmente, el EAT debe establecer y apoyar a la correspondiente organizaci\'{o}n de Product Owners a trav\'{e}s de la asociaci\'{o}n de los POs y sus funciones escaladas de PO con su correspondiente c\'{i}rculo de SoSs. Estos equipos de Product Owners y \textit{stakeholders }clave se conocen como \textbf{MetaScrums}.

\noindent 
\paragraph{Salidas / Resultados del Ciclo de Scrum Master}

\noindent 

\noindent La organizaci\'{o}n de SM (SoS, SoSoS y EAT) trabaja como un todo para completar los otros componentes del Ciclo de Scrum Master: \textbf{Mejora Continua y Eliminaci\'{o}n de Impedimentos, Coordinaci\'{o}n entre equipos e Implementaci\'{o}n}.

\noindent Los objetivos de la Mejora Continua y Eliminaci\'{o}n de Impedimentos son:

\begin{enumerate}
\item  identificar impedimentos y replantearlos como oportunidades,

\item  mantener un ambiente saludable y estructurado para priorizar y eliminar impedimentos y luego verificar las mejoras resultantes, y

\item  garantizar visibilidad en la organizaci\'{o}n para efectuar el cambio.
\end{enumerate}

\noindent 

\noindent Los objetivos de la Coordinaci\'{o}n entre equipos son:

\begin{enumerate}
\item  coordinar procesos similares a trav\'{e}s de m\'{u}ltiples equipos relacionados,

\item  mitigar las dependencias entre equipos para garantizar que no se conviertan en impedimentos y

\item  mantener alineadas las normas y pautas de los equipos para un resultado congruente.
\end{enumerate}

\noindent Dado que el objetivo del SoS es funcionar como un equipo de \textit{release}, la implementaci\'{o}n del producto est\'{a} dentro de su alcance, mientras que lo que contiene cualquier \textit{release }est\'{a} dentro del alcance de los Product Owners. Por lo tanto, los objetivos de Implementaci\'{o}n son:

\begin{enumerate}
\item  entregar un flujo congruente de producto valioso terminado para los clientes,

\item  integrar el trabajo de diferentes equipos en un solo producto y

\item  garantizar la alta calidad de la experiencia del cliente.
\end{enumerate}

\noindent 

\noindent 

\noindent 

\noindent 

\noindent 
\section{Ciclo del Product Owner}

\noindent 
\subsection{Coordinar el "Qu\'{e}" -  El MetaScrum}

\noindent 

\noindent Un grupo de Product Owners que necesita coordinar un \textit{backlog }compartido\textit{ }que alimenta una red de equipos, son ellos mismos un equipo denominado \textbf{MetaScrum}. Para cada SoS hay un MetaScrum asociado. Un MetaScrum alinea las prioridades de los equipos en una \'{u}nica direcci\'{o}n, para que ellos puedan coordinar sus \textit{backlogs }y construye un alineamiento com\'{u}n con los \textit{stakeholders }para que apoyen el \textit{Backlog}.  El Product Owner del equipo  es responsable de la composici\'{o}n y la priorizaci\'{o}n del \textit{backlog }del equipo y puede tomar \'{i}tems del \textit{backlog }compartido del metascrum para el \textit{backlog }del equipo o generar un \textit{backlog }independiente a su discreci\'{o}n.

\noindent 

\noindent Los MetaScrums mantienen una versi\'{o}n escalada de Refinamiento del Backlog, la \textbf{Reuni\'{o}n Escalada de Refinamiento del Backlog}.

\begin{enumerate}
\item  Cada Product Owner (o miembro de su equipo) debe asistir

\item  Este evento es el foro para que los L\'{i}deres de la organizaci\'{o}n, \textit{Stakeholders }u otros Clientes expresen sus preferencias
\end{enumerate}

\noindent 

\noindent Este evento se realiza tantas veces como sea necesario, al menos una vez por Sprint, para asegurar un Backlog listo. 

\noindent 

\noindent Las principales funciones del MetaScrum son:

\begin{enumerate}
\item  crear una visi\'{o}n general para el producto y hacerla visible para la organizaci\'{o}n,

\item  construir un alineamiento com\'{u}n con los \textit{stakeholders }para asegurar el apoyo para la implementaci\'{o}n del Backlog,

\item  generar un \textit{backlog }\'{u}nico y priorizado, que asegurar\'{a} evitar la duplicaci\'{o}n de trabajo.

\item  crear una m\'{i}nima "Definici\'{o}n de Hecho" uniforme que aplique a todos los equipos en el SoS,

\item  eliminar las dependencias levantadas por el SoS,

\item  generar un Plan de Release coordinado y

\item  decidir y controlar las m\'{e}tricas que den \textit{insight }del producto.
\end{enumerate}

\noindent 

\noindent Los MetaScrums, al igual que los SoS, funcionan como equipos Scrum. Como tales, necesitan tener a alguien que act\'{u}e como SM y mantenga al equipo en el camino correcto durante las discusiones. Tambi\'{e}n necesitan una sola persona que sea responsable de coordinar la generaci\'{o}n de un solo Product Backlog para todos los equipos cubiertos por el MetaScrum. Esta persona es designada como \textbf{Chief Product Owner}.

\noindent \textbf{\underbar{}}

\noindent \eject 

\noindent 
\paragraph{El Chief Product Owner (CPO)}

\noindent 

\noindent Por medio de los MetaScrums, los Chief Product Owners coordinan las prioridades entre los Product Owners que trabajan con cada uno de los equipos. Los CPO alinean las prioridades del Backlog con los \textit{stakeholders }y las necesidades de los clientes. Al igual que un SoSM, pueden ser un equipo individual que elige ejercer el rol de Product Owner, o puede haber una persona espec\'{i}ficamente dedicada a este rol. Sus principales responsabilidades son las mismas que las de un Product Owner regular, pero a escala:

\noindent 

\begin{enumerate}
\item  Definir una visi\'{o}n estrat\'{e}gica para todo el producto.

\item  Crear un backlog \'{u}nico de valor priorizado a ser entregado por todos los equipos.

\begin{enumerate}
\item  Estos ser\'{i}an items de Product Backlog m\'{a}s grandes que los de un equipo de un PO.
\end{enumerate}

\item  Trabajar estrechamente con su SoSM asociado para que el Plan de Release generado por el equipo de MetaScrum se pueda implementar de manera eficiente.

\item  Monitorear la retroalimentaci\'{o}n\textit{ }de los clientes sobre el producto y ajustar el \textit{backlog }en consecuencia.
\end{enumerate}

\noindent 
\paragraph{Escalar el MetaScrum}

\noindent 

\noindent As\'{i} como los SoS pueden convertirse en SoSoS, los MetaScrums tambi\'{e}n pueden ampliarse por el mismo mecanismo. No hay un t\'{e}rmino espec\'{i}fico asociado con estas unidades ampliadas, ni los CPO de ellos tienen t\'{i}tulos ampliados espec\'{i}ficos. Alentamos a cada organizaci\'{o}n a crear sus propios t\'{e}rminos. Para los siguientes diagramas, hemos elegido agregar un "Chief" al t\'{i}tulo de esos Product Owners a medida que se ampl\'{i}an.

\noindent 

\noindent Diagramas de ejemplo: 

\noindent 

\begin{tabular}{|p{2.1in}|p{2.1in}|} \hline 
\includegraphics*[width=2.35in, height=2.44in, keepaspectratio=false, trim=0.72in 0.43in 3.18in 0.00in]{image17}\newline MetaScrum de 5 Equipos & \includegraphics*[width=2.67in, height=2.49in, keepaspectratio=false, trim=3.48in 0.37in 0.10in 0.00in]{image18}\newline MetaScrum de 25 Equipos \\ \hline 
\end{tabular}



\noindent \textbf{NOTA}: Como se mencion\'{o} anteriormente, estos pent\'{a}gonos representan los equipos Scrum y MetaScrum de tama\~{n}o ideal. Estos diagramas son solo ejemplos, su diagrama organizativo puede variar considerablemente.

\noindent 
\subsection{El MetaScrum Ejecutivo (EMS)}

\noindent Los MetaScrums permiten un dise\~{n}o de red de Product Owners que es infinitamente escalable junto con sus SoS asociados. El MetaScrum para toda la organizaci\'{o}n Agile es el MetaScrum Ejecutivo. El EMS (por sus siglas en ingl\'{e}s) posee la visi\'{o}n organizacional y establece la prioridades estrat\'{e}gicas para toda la organizaci\'{o}n, alineando a todos los equipos en pos de objetivos comunes.

\noindent 

\noindent Diagrama de ejemplo de un EMS que coordina 5 grupos de 25 equipos:

\noindent \includegraphics*[width=5.56in, height=4.42in, keepaspectratio=false]{image19}

\noindent 
\paragraph{Salidas / Resultados de la Organizaci\'{o}n de Product Owners}

\noindent 

\noindent La organizaci\'{o}n de Product Owners (varios MetaScrums, los CPO y el MetaScrum Ejecutivo) trabajan como un todo para satisfacer los componentes del Ciclo de Product Owner: \textbf{Visi\'{o}n estrat\'{e}gica, Priorizaci\'{o}n del backlog, Descomposici\'{o}n y Refinamiento del Backlog, y Planificaci\'{o}n de Release}.

\noindent 

\noindent Los objetivos de establecer una Visi\'{o}n Estrat\'{e}gica son:

\begin{enumerate}
\item  alinear claramente toda la organizaci\'{o}n con una visi\'{o}n compartida de futuro,

\item  explicar convincentemente de manera articulada el por qu\'{e} de la existencia de la organizaci\'{o}n,

\item  describir qu\'{e} har\'{a} la organizaci\'{o}n para aprovechar los activos clave en pos de su misi\'{o}n y

\item  responder a los repentinos cambios de las condiciones del mercado.
\end{enumerate}

\noindent 

\noindent Los objetivos de la Priorizaci\'{o}n del Backlog son:

\begin{enumerate}
\item  identificar un orden claro de los productos, funcionalidades y servicios que se entregar\'{a}n,

\item  reflejar la creaci\'{o}n de valor, mitigaci\'{o}n de riesgos y dependencias internas en el orden del Backlog y

\item  priorizar las iniciativas estrat\'{e}gicas a lo largo de toda la organizaci\'{o}n Agile antes de la Descomposici\'{o}n y Refinamiento del Backlog.
\end{enumerate}

\noindent 

\noindent Los objetivos de la Descomposici\'{o}n y Refinamiento del Backlog son:

\begin{enumerate}
\item  dividir productos y proyectos complejos en elementos funcionales independientes que puedan ser completados por un equipo en un Sprint,

\item  capturar y analizar los requisitos emergentes y la retroalimentaci\'{o}n de los clientes y

\item  asegurar que todos los \'{i}tems del Backlog {}{}est\'{e}n realmente ``listos'' para que puedan ser tomados por los equipos individuales.
\end{enumerate}

\noindent 

\noindent Los objetivos de la Planificaci\'{o}n de Releases  son:

\begin{enumerate}
\item  prever la entrega de funcionalidades y capacidades clave,

\item  comunicar las expectativas de entregas a los \textit{stakeholders} y

\item  actualizar la priorizaci\'{o}n, seg\'{u}n sea necesario.
\end{enumerate}

\noindent 
\section{Conectar los ciclos de PO/SM}

\noindent 
\subsection{Entender la Retroalimentaci\'{o}n}

\noindent 

\noindent El componente de \textbf{Retroalimentaci\'{o}n }es el segundo punto donde se conectan los ciclos de PO y SM. La retroalimentaci\'{o}n del producto impulsa la mejora continua ajustando el Product Backlog mientras la retroalimentaci\'{o}n del Release  impulsa la mejora continua mediante el ajuste de los mecanismos de implementaci\'{o}n. Los objetivos de obtener y analizar la retroalimentaci\'{o}n son:

\begin{enumerate}
\item  validar nuestras suposiciones,

\item  comprender c\'{o}mo los clientes usan el producto e interact\'{u}an con \'{e}l,

\item  obtener ideas para nuevas caracter\'{i}sticas y funcionalidades,

\item  definir mejoras sobre las funcionalidades existentes,

\item  actualizar el progreso hacia la finalizaci\'{o}n del producto/proyecto para refinar la planificaci\'{o}n de \textit{releases }y el alineamiento con los \textit{stakeholders}, e

\item  identificar mejoras para  los mecanismos y m\'{e}todos de implementaci\'{o}n.
\end{enumerate}

\noindent 

\noindent 

\noindent \eject 

\noindent 
\subsection{M\'{e}tricas y Transparencia}

\noindent 

\noindent La transparencia radical es esencial para que Scrum funcione \'{o}ptimamente, pero solo es posible en una organizaci\'{o}n que ha adoptado los valores de Scrum. Le da a la organizaci\'{o}n la capacidad de evaluar honestamente su progreso e inspeccionar y adaptar sus productos y procesos. Esta es la base de la naturaleza emp\'{i}rica de Scrum como se establece en la Gu\'{i}a Scrum. 

\noindent 

\noindent Tanto el ciclo de SM como el ciclo de PO requieren m\'{e}tricas que ser\'{a}n decididas por separado por las organizaciones de SM y PO. Las m\'{e}tricas pueden ser \'{u}nicas tanto para las organizaciones espec\'{i}ficas como para las funciones espec\'{i}ficas dentro de esas organizaciones. Scrum@Scale no requiere ning\'{u}n conjunto espec\'{i}fico de m\'{e}tricas, pero sugiere que, como m\'{i}nimo, la organizaci\'{o}n deber\'{i}a medir:

\noindent 

\begin{enumerate}
\item  Productividad - Ej.: cambios en la cantidad de Producto funcionando entregado por Sprint.

\item  Entrega de valor - Ej.: valor del negocio por unidad de esfuerzo del equipo.

\item  Calidad - Ej.: cantidad de defectos o tiempo de inactividad del servicio.

\item  Sostenibilidad - Ej.:  Felicidad del equipo.
\end{enumerate}

\noindent 

\noindent Los objetivos de tener M\'{e}tricas y Transparencia son:

\begin{enumerate}
\item  proporcionar el ambiente adecuado a todos los responsables de tomar decisiones (incluidos los miembros del equipo) para que tomen buenas decisiones,

\item  acortar los ciclos de retroalimentaci\'{o}n tanto como sea posible para evitar la correcci\'{o}n excesiva y

\item  requerir el m\'{i}nimo esfuerzo adicional de equipos, \textit{stakeholders }o l\'{i}deres.
\end{enumerate}

\noindent 
\subsection{Algunas notas sobre el dise\~{n}o organizacional}

\noindent 

\noindent La naturaleza de ser aplicable a toda escala de Scrum@Scale permite que el dise\~{n}o de la organizaci\'{o}n est\'{e} basado en componentes, al igual que el \textit{framework }en s\'{i}. Esto permite reajustar o rebalancear  los equipos en respuesta al mercado. A medida que una organizaci\'{o}n crece, capturar los beneficios de tener equipos distribuidos puede ser importante. Algunas organizaciones consiguen un talento que de otra manera no lograr\'{i}an alcanzar y pueden expandirse y contratar, seg\'{u}n lo que necesiten, por desarrollo tercerizado. Scrum@Scale muestra c\'{o}mo hacerlo evitando largos retrasos, problemas de comunicaci\'{o}n y baja calidad, permitiendo una escalabilidad lineal tanto en tama\~{n}o como en distribuci\'{o}n global\footnote{\ Sutherland,\ Jeff\ and\ Schoonheim,\ Guido\ and\ Rustenburg,\ Eelco\ and\ Rijk,\ Maurits,\ ``Fully\ distributed\ scrum:\ The\ secret\ sauce\ for\ hyperproductive\ offshored\ development\ teams'',\ AGILE'08.\ Conference,\ IEEE:\ 339-344,\ 2008}.

\noindent 

\noindent 

\noindent 

\noindent 

\noindent Diagramas de ejemplo:

\begin{tabular}{|p{2.1in}|p{2.1in}|} \hline 
\includegraphics*[width=2.65in, height=2.47in, keepaspectratio=false, trim=0.00in 0.28in 3.39in 0.00in]{image20}\newline 5 SoS con 2, 3, 4 y 2x5 equipos & \includegraphics*[width=2.91in, height=2.43in, keepaspectratio=false, trim=3.13in 0.32in 0.00in 0.00in]{image21}\newline 3 SoSoS con 10, 13 y 15 equipos \\ \hline 
\end{tabular}



\noindent \includegraphics*[width=6.19in, height=3.41in, keepaspectratio=false]{image22}

\noindent En este diagrama organizacional, los \textbf{Equipos de Conocimiento e Infraestructura} representan equipos virtuales de especialistas escasos para ser asignados a  cada equipo. Ellos se coordinan con los equipos Scrum como grupo mediante acuerdos de nivel de servicio en los cuales los requerimientos fluyen a trav\'{e}s de un PO para cada especialidad que los convierte en un Backlog ordenado y transparente. Una nota importante es que estos equipos NO son silos de individuos que participan juntos (por eso est\'{a}n representados como pent\'{a}gonos huecos) sino que sus miembros de equipo participan con los equipos reales de Scrum, pero componen este Scrum virtual propio con el prop\'{o}sito de difundir el \textit{backlog }y mejorar el proceso.

\noindent 

\noindent \textbf{Relaciones con el cliente, Legal / Compliance y  Operaciones} se incluyen aqu\'{i} ya que son partes necesarias de las organizaciones y existir\'{a}n como equipos Scrum independientes en los que todos los dem\'{a}s pueden confiar.

\noindent Una nota final sobre la representaci\'{o}n de EAT \& EMS: en este diagrama, se muestran superpuestas ya que 2 miembros participan en ambos equipos. En organizaciones o implementaciones muy peque\~{n}as, el EAT \& EMS pueden estar completamente formados por los mismos miembros.

\noindent 
\section{Nota final}

\noindent 

\noindent Scrum@Scale est\'{a} dise\~{n}ada para escalar la productividad, para hacer que toda la organizaci\'{o}n haga el doble de trabajo en la mitad de tiempo con mayor calidad y en un ambiente de trabajo mejorado en forma significativa. Las grandes organizaciones que implementen adecuadamente el \textit{framework }pueden reducir el costo de sus productos y servicios mientras mejoran la calidad y la innovaci\'{o}n.

\noindent 

\noindent Scrum@Scale est\'{a} dise\~{n}ado para saturar una organizaci\'{o}n con Scrum. Todos los equipos, incluidos los L\'{i}deres, Recursos Humanos, Legales, Consultor\'{i}a y Capacitaci\'{o}n, y equipos de producto y servicio, implementan el mismo estilo de Scrum mientras optimizan y mejoran la organizaci\'{o}n.

\noindent 

\noindent Si est\'{a} bien implementado, Scrum puede gestionar una organizaci\'{o}n completa.

\noindent 
\section{Agradecimientos}

\noindent 

\noindent Reconocemos a IDX por la creaci\'{o}n de Scrum de Scrums que permiti\'{o} por primera vez escalar Scrum a cientos de equipos\footnote{\ J.\ Sutherland,\ "Agile\ Can\ Scale:\ Inventing\ and\ Reinventing\ Scrum\ in\ Five\ Companies,"\ (Cutter\ IT\ Journal\ 14(12):\ 5-11,\ 2001).}, a PatientKeeper por la creaci\'{o}n de MetaScrum\footnote{\ J.\ Sutherland,\ ``Future\ of\ Scrum:\ Parallel\ Pipelining\ of\ Sprints\ in\ Complex\ Projects,''\ (Agile\ `05,\ Denver,\ 2005).}, que permiti\'{o} la implementaci\'{o}n r\'{a}pida de productos innovadores, y a OpenView Venture Partners por escalar Scrum a la totalidad de la organizaci\'{o}n\footnote{\ J.\ Sutherland\ and\ I.\ Altman,\ ``Take\ No\ Prisoners:\ How\ a\ Venture\ Capital\ Group\ Does\ Scrum,''\ (Agile\ `09,\ Chicago,\ 2009).}. Valoramos los aportes de Intel con m\'{a}s de 25.000 personas que hacen Scrum y nos ense\~{n}aron que "nada escala" a excepci\'{o}n de una arquitectura aplicable a toda escala, y a SAP que tiene la organizaci\'{o}n m\'{a}s grande de equipos Scrum de productos que nos ense\~{n}\'{o} que el involucramiento del \textit{management }en el MetaScrum es esencial para lograr que 2.000 equipos Scrum trabajen juntos. 

\noindent 

\noindent Los \textit{coaches }y entrenadores Agile que implementan estos conceptos en Amazon, GE, 3M, Toyota, Spotify y muchas otras compa\~{n}\'{i}as que trabajan con Jeff Sutherland han sido de gran ayuda para probar estos conceptos en una amplia variedad de compa\~{n}\'{i}as en diferentes \'{a}reas.

\noindent 

\noindent Y, finalmente, Avi Schneier y Alex Sutherland han tenido un valor incalculable en la formulaci\'{o}n y edici\'{o}n de este documento.


\end{document}

